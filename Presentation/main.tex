\documentclass[10pt, xcolor=x11names,compress]{beamer}
\usepackage{tabulary}
\usepackage{booktabs}
\usepackage{float}
\usepackage{graphicx}
\usepackage{siunitx}
\usepackage{hyperref}
\usepackage{multicol}
\usepackage{biblatex}
\usepackage[spanish, mexico]{babel}
% Start custom color definition
\definecolor{gnuhealth}{RGB}{21,110,182}
% End custom color definition

\usecolortheme{spruce}
\useoutertheme{infolines}
\usefonttheme[onlymath]{serif}
\setbeamertemplate{headline}[default]
\setbeamertemplate{navigation symbols}{}
\mode<beamer>{\setbeamertemplate{blocks}[rounded]}
\setbeamercovered{transparent}
\setbeamercolor{block body}{use=structure, fg=gnuhealth, bg=gnuhealth!20}
\setbeamercolor{block title}{use=structure, fg=gray!20, bg=gnuhealth}
\setbeamercolor{itemize item}{fg=black}
\setbeamercolor{itemize subitem}{fg=gray}
\setbeamercolor{itemize subsubitem}{fg=black!20}

% Start custom palette
\setbeamercolor{palette primary}{fg=gnuhealth,bg=gray!15}
\setbeamercolor{palette secondary}{fg=gnuhealth,bg=gnuhealth!15}
\setbeamercolor{palette tertiary}{fg=gnuhealth!10,bg=gnuhealth}
\setbeamercolor{structure}{fg=gnuhealth}
\setbeamercolor{frametitle}{bg=gnuhealth!10}
% End custom palette

% Start commands to omit miniframes inclusion of section pages
\makeatletter
\let\beamer@writeslidentry@miniframeson=\beamer@writeslidentry%
\def\beamer@writeslidentry@miniframesoff{%
  \expandafter\beamer@ifempty\expandafter{\beamer@framestartpage}{}% does not happen normally
  {%else
    % removed \addtocontents commands
    \clearpage\beamer@notesactions%
  }
}
\newcommand*{\miniframeson}{\let\beamer@writeslidentry=\beamer@writeslidentry@miniframeson}
\newcommand*{\miniframesoff}{\let\beamer@writeslidentry=\beamer@writeslidentry@miniframesoff}
\makeatother

% End commands to omit miniframes inclusion of section pages


% Start presentation section pages
\AtBeginSection[]{
    {
      \miniframesoff
      \setbeamercolor{background canvas}{bg=gnuhealth}
      \begin{frame}[plain]
      \vfill
      \centering
        \usebeamerfont{title}\color{gnuhealth!10}{ \insertsectionhead}\par
      \vfill
      \end{frame}
      \miniframeson
    }
}
% End presentation section pages

% Start config miniframes, header and footer
\makeatletter\setbeamertemplate{footline}
{  
\leavevmode%  
\hbox{%  
\begin{beamercolorbox}[wd=.333333\paperwidth,ht=2.25ex,dp=1ex,center]{author in head/foot}%    
\usebeamerfont{author in head/foot}
\insertshortauthor%~~\beamer@ifempty{\insertshortinstitute}{}
 \end{beamercolorbox}%  
 \begin{beamercolorbox}[wd=.333333\paperwidth,ht=2.25ex,dp=1ex,center]{institute in head/foot}%    
 \usebeamerfont{title in head/foot}\insertshortinstitute  
 \end{beamercolorbox}%  
 \begin{beamercolorbox}[wd=.333333\paperwidth,ht=2.25ex,dp=1ex,right]{date in head/foot}%    
 \usebeamerfont{date in head/foot}\insertshortdate{}\hspace*{4em}
 \insertframenumber{} / \inserttotalframenumber\hspace*{2ex}   
 \end{beamercolorbox}}%  
 \vskip0pt%
 }
\makeatother 
\useoutertheme[footline=empty, subsection=false]{miniframes}
% End config miniframes, header and footer

% Start presentation metadata
\author[DM-6213]{Rafael Contreras Agudelo \\ Edwyn Guzmán \\ Yarima Contreras Blanco}
\title{Introducción a Miner\'ia de datos}
\institute[\textsl{Introducci\'on a miner\'ia de datos}]{
    6213 - Miner\'ia de datos \\ \color{black!40} Facultad de ciencias \\ Universidad Central de Venezuela
    \medskip
}
\date{\today}
\bibliography{reference}
% End presentation metadata

% Start util snippets
% Having placed a \begin{frame}[label=Rocky] you can use \hyperlink{Rocky}{\beamergotobutton{go}} to put a button to get you there

% Insert image:
% \begin{figure}
%     \centering
%     \includegraphics[width=0.3\textwidth]{Figure1.jpg}
%     \caption{His name is Rocky}
%     \label{fig:enter-label}
% \end{figure}

% End util snippets

\begin{document}

% Start titlepage
\begin{frame}[plain]
\centering
\includegraphics[width=0.12\textwidth]{logos/universidad.png}
\hspace{0.71\textwidth}
\includegraphics[width=0.12\textwidth]{logos/facultad.jpg}
\vspace{0.10\textwidth}
\titlepage
\end{frame}
% End titlepage


% Start Table of content
\setbeamertemplate{section in toc}[circle]
\begin{frame}
\frametitle{Contenido}
\begin{columns}[c]
\column{.5\textwidth}
\setlength{\parskip}{2ex}
 \tableofcontents
\column{.45\textwidth}

\color{black!40}\emph{
"Notoriamente no hay clasificación del universo que no sea arbitraria y conjetural."\footnote{El idioma analítico de John Wilkins - Jorge Luis Borges}
}\\~
\color{black}

\end{columns}
\end{frame}
% End Table of content

% Start body of the presentation
\section{Objetivo de la Miner\'ia de datos}
\begin{frame}{Conceptos b\'asicos}
\begin{description}
    \item[Definici\'on] Es el proceso de \textbf{descubrir conocimiento} o \textbf{patrones} \\a partir de \textbf{datos}.\cite{DBLP:reference/db/2018}(*)
\end{description}
\end{frame}
%-----
\begin{frame}{Conceptos básicos}
\begin{description}
\item[Definición]<1-> Es el proceso de \textbf{descubrir conocimiento} o \textbf{patrones} \\a partir de \textbf{datos}.\cite{DBLP:reference/db/2018}(*)
\item[Datos]<1->Abstracci\'on representada estructuradamente de uno o varios hechos.
\end{description}
%-----
\begin{figure}
    \centering
    \includegraphics<2>[width=0.37\textwidth]{figs/data-1.png}
    \label{fig:data-1}
\end{figure}

\begin{figure}
    \centering
    \includegraphics<3>[height=0.4\textwidth]{figs/data-2.png}
    \label{fig:data-1}
\end{figure}
\end{frame}
%-----
\section{Definici\'on del problema}
\begin{frame}{Definici\'on del problema}
\begin{description}
\item[Definición]<1-> Es el proceso de \textbf{descubrir conocimiento} o \textbf{patrones} \\a partir de \textbf{datos}.\cite{DBLP:reference/db/2018}(*)

\item[Datos]<1-> Abstracci\'on representada estructuradamente de uno o varios hechos.
\item[Conocimiento]<1-> Información implicita, desconocida y potencialmente útil observadas en datos. \cite{aimag-v13i3-1011}
\end{description}
\vspace{2em}
 \begin{columns}[c]<2>
\column{.5\textwidth}
\textbf{Ejemplos de datos}
 \begin{itemize}
     \item \# b\'usquedas de 'IA' en tiempo
     \item Tweets de muestra de inter\'es
     \item Series vistas por usuarios
     \item<0> ${datos}_i $
 \end{itemize}
\column{.45\textwidth}
\textbf{Ejemplos de conocimiento}
 \begin{itemize}
     \item Inter\'es acad\'emico en IA
     \item Opini\'on p\'ublica*
     \item \textsl{"You may also like"}
     \item<0> ${conocimiento}_{ij}$
 \end{itemize}
\end{columns}
\end{frame}
%-----
\begin{frame}{Tipos de conjuntos de datos}
\begin{columns}
    \column{0.5\textwidth}
    \begin{itemize}
        \item<1> Datos multi-dimensionales
        \item<1> Datos transaccionales
        \item<1> Datos temporales
        \item<0> Lenguaje natural
        \item<0> Audio
        \item<0> Im\'agenes
        \item<0> Datos espaciales
        \item<0> Datos orientados a gr\'afos
    \end{itemize}
    \column{0.45\textwidth}<1>
    \begin{table}[c]
     \centering
     \begin{tabular}{c|c|c|c}
         \textbf{ID} & \textbf{${atributo}_1$} & ... & \textbf{${atributo}_i$} \\
         \hline
          1 & ${val}_{1,1}$ & ... & ${val}_{1,i}$ \\
          2 & ${val}_{2,1}$ & ... & ${val}_{2,i}$ \\
     \end{tabular}
     \caption{Estrutura ejemplo}
     \label{tab:websites_visits}
 \end{table}
 \centering
 $type({atributo}_i) \in [bool, enum, number, string, date]$
\end{columns}
\end{frame}
%-----
\begin{frame}{Tipos de conjuntos de datos}
\begin{columns}
    \column{0.5\textwidth}
\begin{itemize}
    \item<0> Datos multi-dimensionales
    \item<0> Datos transaccionales
    \item<0> Datos temporales
    \item<1> Lenguaje natural
    \item<1> Audio
    \item<1> Im\'agenes
    \item<0> Datos espaciales
    \item<0> Datos orientados a gr\'afos
\end{itemize}
    \column{0.45\textwidth}<1>
    \begin{figure}
    \centering
    \includegraphics[width=\textwidth]{figs/datasets-1.png}
    \label{fig:datasets-1}
\end{figure}
\end{columns}
\end{frame}
%-----
\section[Taxonom\'ia de t\'ecnicas]{Taxonom\'ia de t\'ecnicas de MD}
\begin{frame}{Tipos de conocimiento a extraer\cite{han2012mining}}
\begin{columns}
    \column{0.5\textwidth}
\begin{itemize}
    \item<1> Tareas descriptivas
    \begin{itemize}
        \item<1> Caracterizaci\'on de datos
        \item<0> Discriminaci\'on de datos
    \end{itemize}
    \item<0> Tareas predictivas
    \begin{itemize}
        \item Clasificaci\'on
        \item Regresi\'on
    \end{itemize}
    \item<0> Agrupaci\'on
    \item<0> Detecci\'on de anomal\'ias
    \item<0> Asociaciones
\end{itemize}
    \column{0.45\textwidth}<1> 
        \begin{figure}[c]
    \centering
    \includegraphics[width=.8\textwidth]{figs/descr-1.png}
    \label{fig:descr-1}
\end{figure}
 \centering
\end{columns}
\end{frame}
%-----
\begin{frame}{Tipos de conocimiento a extraer}
\begin{columns}
    \column{0.5\textwidth}
\begin{itemize}
    \item<1> Tareas descriptivas\footnote{\color{gnuhealth!60}kaggle.com/search?q=eda}
    \begin{itemize}
        \item<0> Caracterizaci\'on de datos
        \item<1> Discriminaci\'on de datos
    \end{itemize}
    \item<0> Tareas predictivas
    \begin{itemize}
        \item Clasificaci\'on
        \item Regresi\'on
    \end{itemize}
    \item<0> Agrupaci\'on
    \item<0> Detecci\'on de anomal\'ias
    \item<0> Asociaciones
\end{itemize}
    \column{0.45\textwidth}<1> 
        \begin{figure}[c]
    \centering
    \includegraphics[width=.8\textwidth]{figs/descr-2.png}
    \label{fig:descr-2}
\end{figure}
 \centering
\end{columns}
\end{frame}
%-----
\begin{frame}{Tipos de conocimiento a extraer}
\begin{columns}
    \column{0.5\textwidth}
\begin{itemize}
    \item<0> Tareas descriptivas
    \begin{itemize}
        \item Caracterizaci\'on de datos
        \item Discriminaci\'on de datos
    \end{itemize}
    \item<1> Tareas predictivas
    \begin{itemize}
        \item Clasificaci\'on
        \item Regresi\'on
    \end{itemize}
    \item<0> Agrupaci\'on
    \item<0> Detecci\'on de anomal\'ias
    \item<0> Asociaciones
\end{itemize}
    \column{0.45\textwidth}<1>
        \begin{table}[c]
     \centering
     \begin{tabular}{c|c|c|c}
         \textbf{ID} & ... & \textbf{${atributo}_i$} & \textsl{objetivo}\\
         \hline
          1 & ... & ${val}_{1,i}$ & ${obj}_{1}$\\
          2 & ... & ${val}_{2,i}$ & ${obj}_{2}$\\
     \end{tabular}
     \caption{Estrutura ejemplo}
     \label{tab:websites_visits}
 \end{table}
 \centering
 $type({obj}_i) = enum \rightarrow$ clasificaci\'on
 $type({obj}_i) = number \rightarrow$ regresi\'on
\end{columns}
\end{frame}
%-----
\begin{frame}{Tipos de conocimiento a extraer}
\begin{columns}
    \column{0.45\textwidth}
\begin{itemize}
    \item<0> Tareas descriptivas
    \begin{itemize}
        \item Caracterizaci\'on de datos
        \item Discriminaci\'on de datos
    \end{itemize}
    \item<1> Tareas predictivas
    \begin{itemize}
        \item Clasificaci\'on
        \item Regresi\'on
    \end{itemize}
    \item<0> Agrupaci\'on
    \item<0> Detecci\'on de anomal\'ias
    \item<0> Asociaciones
\end{itemize}
    \column{0.5\textwidth}<1>
    \textbf{Ejemplo de aplicaciones}
    \begin{itemize}
        \item Determinar la categoría de una p\'agina web
        \item Identificar el sentimiento predominante de un tweet
        \item Categorizar candidatos a cr\'editos bancarios
        \item Estimar el costo de un inmueble
        \item Estimar el crecimiento de una compa\~n\'ia
        \item Estimar la popularidad de una canci\'on en spotify\footnote{\color{gnuhealth!60}\href{https://hf.co/datasets/maharshipandya/spotify-tracks-dataset}{hf.co/datasets/maharshipandya/spotify-tracks-dataset}}
    \end{itemize}
\end{columns}
\end{frame}
%-----
\begin{frame}{Tipos de conocimiento a extraer}
\begin{columns}
    \column{0.5\textwidth}
\begin{itemize}
    \item<0> Tareas descriptivas
    \begin{itemize}
        \item Caracterizaci\'on de datos
        \item Discriminaci\'on de datos
    \end{itemize}
    \item<0> Tareas predictivas
    \begin{itemize}
        \item Clasificaci\'on
        \item Regresi\'on
    \end{itemize}
    \item<1> Agrupaci\'on
    \item<0> Detecci\'on de anomal\'ias
    \item<0> Asociaciones
\end{itemize}
    \column{0.45\textwidth}<1> 
        \begin{figure}[c]
    \centering
    \includegraphics[width=.8\textwidth]{figs/agrup-1.png}
    \label{fig:agrup-1}
\end{figure}
 \centering
\end{columns}
\end{frame}
%-----
\begin{frame}{Tipos de conocimiento a extraer}
\begin{columns}
    \column{0.5\textwidth}
\begin{itemize}
    \item<0> Tareas descriptivas
    \begin{itemize}
        \item Caracterizaci\'on de datos
        \item Discriminaci\'on de datos
    \end{itemize}
    \item<0> Tareas predictivas
    \begin{itemize}
        \item Clasificaci\'on
        \item Regresi\'on
    \end{itemize}
    \item<1> Agrupaci\'on
    \item<0> Detecci\'on de anomal\'ias
    \item<0> Asociaciones
\end{itemize}
    \column{0.45\textwidth}<1> 
        \begin{figure}[c]
    \centering
    \includegraphics[width=.8\textwidth]{figs/agrup-2.png}
    \label{fig:agrup-2}
\end{figure}
 \centering
\end{columns}
\end{frame}
%-----
\begin{frame}{Tipos de conocimiento a extraer}
\begin{columns}
    \column{0.5\textwidth}
\begin{itemize}
    \item<0> Tareas descriptivas
    \begin{itemize}
        \item Caracterizaci\'on de datos
        \item Discriminaci\'on de datos
    \end{itemize}
    \item<0> Tareas predictivas
    \begin{itemize}
        \item Clasificaci\'on
        \item Regresi\'on
    \end{itemize}
    \item<1> Agrupaci\'on
    \item<0> Detecci\'on de anomal\'ias
    \item<0> Asociaciones
\end{itemize}
\column{0.5\textwidth}<1>
    \textbf{Ejemplo de aplicaciones}
    \begin{itemize}
        \item Segmentaci\'on de clientes
        \item Agrupaci\'on de tweets por semajanza sem\'antica\footnote{\color{gnuhealth!60}\href{https://sbert.net}{sbert.net}}
        \item Agrupar estudiantes con desempe\~nos similares
    \end{itemize}
\end{columns}
\end{frame}
%-----
\begin{frame}{Tipos de conocimiento a extraer}
\begin{columns}
    \column{0.5\textwidth}
\begin{itemize}
    \item<0> Tareas descriptivas
    \begin{itemize}
        \item Caracterizaci\'on de datos
        \item Discriminaci\'on de datos
    \end{itemize}
    \item<0> Tareas predictivas
    \begin{itemize}
        \item Clasificaci\'on
        \item Regresi\'on
    \end{itemize}
    \item<0> Agrupaci\'on
    \item<1> Detecci\'on de anomal\'ias
    \item<0> Asociaciones
\end{itemize}
    \column{0.45\textwidth}<1> 
        \begin{figure}[c]
    \centering
    \includegraphics[width=.8\textwidth]{figs/outl-1.png}
    \label{fig:outl-1}
\end{figure}
 \centering
\end{columns}
\end{frame}
%-----
\begin{frame}{Tipos de conocimiento a extraer}
\begin{columns}
    \column{0.5\textwidth}
\begin{itemize}
    \item<0> Tareas descriptivas
    \begin{itemize}
        \item Caracterizaci\'on de datos
        \item Discriminaci\'on de datos
    \end{itemize}
    \item<0> Tareas predictivas
    \begin{itemize}
        \item Clasificaci\'on
        \item Regresi\'on
    \end{itemize}
    \item<0> Agrupaci\'on
    \item<1> Detecci\'on de anomal\'ias
    \item<0> Asociaciones
\end{itemize}
    \column{0.5\textwidth}<1>
    \textbf{Ejemplo de aplicaciones}
    \begin{itemize}
        \item Detecci\'on de ataques a servidores
        \item Detecci\'on de fraudes bancarios
        \item Identificar las fallas de un servicio de telefonía móvil
    \end{itemize}
\end{columns}
\end{frame}
%-----
\begin{frame}{Tipos de conocimiento a extraer}
\begin{columns}
    \column{0.5\textwidth}
\begin{itemize}
    \item<0> Tareas descriptivas
    \begin{itemize}
        \item Caracterizaci\'on de datos
        \item Discriminaci\'on de datos
    \end{itemize}
    \item<0> Tareas predictivas
    \begin{itemize}
        \item Clasificaci\'on
        \item Regresi\'on
    \end{itemize}
    \item<0> Agrupaci\'on
    \item<0> Detecci\'on de anomal\'ias
    \item<1> Asociaciones
\end{itemize}
    \column{0.45\textwidth}<1> 
        \begin{figure}[c]
    \centering
    \includegraphics[width=\textwidth]{figs/asoc-1.png}
    \label{fig:asoc-1}
\end{figure}
 \centering
\end{columns}
\end{frame}
%-----
\begin{frame}{Tipos de conocimiento a extraer}
\begin{columns}
    \column{0.5\textwidth}
\begin{itemize}
    \item<0> Tareas descriptivas
    \begin{itemize}
        \item Caracterizaci\'on de datos
        \item Discriminaci\'on de datos
    \end{itemize}
    \item<0> Tareas predictivas
    \begin{itemize}
        \item Clasificaci\'on
        \item Regresi\'on
    \end{itemize}
    \item<0> Agrupaci\'on
    \item<0> Detecci\'on de anomal\'ias
    \item<1> Asociaciones
\end{itemize}
    \column{0.5\textwidth}<1>
    \textbf{Ejemplo de aplicaciones}
    \begin{itemize}
        \item \textsl{"Tambi\'en te puede gustar"}
        \item Detecci\'on de ocurrencias probables/improbables
    \end{itemize}
\end{columns}
\end{frame}
%-----
\begin{frame}{Resumen}
\begin{columns}
    \column{.25\textwidth}
    \column{.5\textwidth}
    \begin{block}{Miner\'ia de datos}
    Descubrir conocimiento y/o patrones a partir de datos.*
\end{block}
\column{.25\textwidth}
\end{columns}
    
\end{frame}
%-----
\begin{frame}{Referencias}
\noindent\printbibliography
\end{frame}

\begin{frame}
 \begin{center}
		{\Huge ¡Gracias!}\\
		\bigskip\bigskip % Vertical whitespace
		
		{\LARGE dm-25.ucv.ai}
		
	\end{center}
\end{frame}

\end{document}